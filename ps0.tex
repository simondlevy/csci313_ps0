\documentclass [12pt,psfig]{article}
\usepackage{epsfig}

\baselineskip=15pt minus 1pt
\parskip=3pt plus1pt minus.5pt
\overfullrule=0pt

% maximize page usage
\oddsidemargin -20pt
\evensidemargin -20pt
\marginparwidth 30pt % these gain 53pt width
%\topmargin -20pt       % gains 26pt height
\headheight -10pt      % gains 11pt height
\headsep 1pt         % gains 24pt height
%\footheight 12 pt % cannot be changed as number must fit
\footskip 24pt       % gains 6pt lawall.texheight
\textheight % 528 + 26 + 11 + 24 + 6 + 55 for luck
            680pt
\textwidth % 360 + 53 + 47 for luck
           500pt
% end of page layout changes

\begin{document}

% Alter some LaTeX defaults for better treatment of figures:
% See p.105 of "TeX Unbound" for suggested values.
% See pp. 199-200 of Lamport's "LaTeX" book for details.
%   General parameters, for ALL pages:
\renewcommand{\topfraction}{0.9}	% max fraction of floats at top
\renewcommand{\bottomfraction}{0.8}	% max fraction of floats at bottom
%   Parameters for TEXT pages (not float pages):
\setcounter{topnumber}{2}
\setcounter{bottomnumber}{2}
\setcounter{totalnumber}{4}     % 2 may work better
\setcounter{dbltopnumber}{2}    % for 2-column pages
\renewcommand{\dbltopfraction}{0.9}	% fit big float above 2-col. text
\renewcommand{\textfraction}{0.07}	% allow minimal text w. figs
%   Parameters for FLOAT pages (not text pages):
\renewcommand{\floatpagefraction}{0.7}	% require fuller float pages
    % N.B.: floatpagefraction MUST be less than topfraction !!
\renewcommand{\dblfloatpagefraction}{0.7}	% require fuller float pages

% remember to use [htp] or [htpb] for placement

%\pagestyle{empty}

\begin{flushleft}
\begin{minipage}{1.25in}
\hspace{-0.5in}
\resizebox{\textwidth}{!}{\includegraphics{seal.jpg}}
\end{minipage}
\end{flushleft}

\vspace{-1.25in}

\begin{flushright}
\begin{minipage}{430pt}
{
\begin{flushleft}
{\Large C}{\large OMPUTER} 
{\Large S}{\large CIENCE}
{\Large 313} {\large (Winter Term 2008)\\
\textbf{Theory of Computation}}\\
Prof. Levy\hfill\\
\end{flushleft}
}
\end{minipage}
\end{flushright}

\def\date{\begin{flushright}
\today
\end{flushright}}

\vspace{0.25in}
%\input preamble

\def\set#1{\lbrace #1 \rbrace}

\topmargin 20pt
\indent \indent Date: \underline{Wednesday 14 January 2008}
\indent\indent\indent\indent\indent
Name: \underline{Leeroy Jenkins}
\begin{center}
\rm\LARGE
\bf Assignment\# 0\\
\normalsize
\end{center}
\noindent
\\
\textbf{1.} Draw a DFA that accepts the language of strings having at least one 
$a$.
\begin{figure}[htp]
\centering
\includegraphics[width=1.0\textwidth]{question1.pdf}
\end{figure}

\end{document}
